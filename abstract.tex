The Multi-Order Solar Extreme Ultraviolet Spectrograph (MOSES) sounding rocket was launched from White Sands Missile Range on February 8th, 2006 to capture images of the sun in the \heii \ emission line.
MOSES is a slitless spectrograph that forms images in multiple spectral orders simultaneously using a concave diffraction grating in an effort to measure line profiles over a wide field of view from a single exposure.
Early work on MOSES data showed evidence of solar features composed of neither \heii \ or the nearby \sixi \ spectral lines.
We have built a forward model that uses co-temporal EIT images and the Chianti atomic database to fit synthetic images with known spectra to the MOSES data in order to quantify this additional spectral content.
Our fit reveals a host of dim lines that alone are insignificant but combined contribute a comparable intensity to MOSES images as \sixi.
In total, lines other than \heii\ and \sixi\ contribute approximately 10 percent of the total intensity in the MOSES zero order image.
This additional content, if not properly accounted for, could significantly impact the analysis of MOSES and similar slitless spectrograph data, especially those using a zero order image.